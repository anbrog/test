% NB uses figures from ../figures folder
\documentclass[11pt]{article}
\usepackage{setspace}
\doublespacing
\usepackage{geometry}
\geometry{left=3cm,top=3cm,right=3cm,bottom=3cm}
\usepackage[round, comma, authoryear, sort&compress]{natbib}
\setlength{\bibsep}{5pt}
\usepackage{amsmath}
\usepackage{amsfonts}
\usepackage{graphicx}
\usepackage{rotating} 
\usepackage{caption}
\usepackage{booktabs}
\usepackage{graphicx}
\usepackage{subcaption}
\usepackage{mathtools}
\usepackage{multirow}
\usepackage{tabularx}
\usepackage{pdflscape}
\usepackage{xcolor}
\usepackage{comment}
\usepackage{soul}
\usepackage[utf8]{inputenc}
\usepackage{hyperref} %[hidelinks]
\hypersetup{
	colorlinks,
	linkcolor={red!80!black},
	citecolor={blue!80!black},
	urlcolor={blue!80!black}
}
\usepackage[toc,page]{appendix}
\captionsetup[table]{labelfont={small, bf}, font={small, bf}}
\captionsetup[figure]{labelfont={small, bf}, font={small, bf}}
\captionsetup[subfigure]{font={small, bf}, textfont=normalfont,singlelinecheck=off, justification=centering}

%Additional Packages
\newcommand\fnote[1]{\captionsetup{font=small}\caption*{#1}}
\usepackage{color, colortbl}
\definecolor{Gray}{gray}{0.9}
\definecolor{LightCyan}{rgb}{0.88,1,1}
\renewcommand{\floatpagefraction}{.8}
\usepackage{tikz}
\newcommand{\mytab}[1]{
	\begin{tabular}{@{}c@{}}
		#1
	\end{tabular}
}


\title{\Huge Climate Sentiment\thanks{Alexander Kronies gratefully acknowledges support from the Pension Research Center (PeRCent), and Andreas Br\o gger 
from Center of Financial Frictions (FRIC), grant no. DNRF102.}} %Just changed as its done alphabetically in Finance ;) 
\author{Andreas Br\o gger \hspace{3em} Alexander Kronies \\Copenhagen Business School
}
\date{\today}         % No date for final submission

\begin{document}

\maketitle
\thispagestyle{empty} % removes page number for first page

\begin{abstract}

\noindent Contrary to current evidence that investing ethically does not matter for investors, we find that ethical investment does matter for investors when sentiment is high [in good times]. [We further show that climate sentiment tends to be high in good times]. We examine excess returns for a stock portfolio which is long in high ESG-scoring companies and short in low scoring companies. The results are robust to controlling for risk using the Fama French three and five factor models, and the Carhart four factor model, as well as different horizons for the good time indicator, and different portfolio sizes. Using an event-study approach, we further find that a loss in ESG scores matters for performance at a risk-adjusted level, whereas an equal increase does not. We exploit the growing data on Environmental Social Governance (ESG) scores of companies, as well as a sentiment measure from Google Trends.  These results have implications for the understanding of the willingness of investors to forgo returns for ethical investments; as well as if and when CEOs should focus on their sustainable policies.
\end{abstract}

\noindent \textbf{Keywords}: Sustainable investing, ESG, Equities, Event Studies.\\
\noindent \textbf{JEL Codes}: G12, G14, Q5, G11.

\clearpage
\setcounter{footnote}{0}
\renewcommand{\thefootnote}{\arabic{footnote}}
\setcounter{page}{1}


\section{Introduction}
Over the last two decades Environmental Social Governance (ESG) scores have been growing. There has been reports that ethical companies out-perform regular companies [source eg FT], alluding that ESG affects firm performance. This paper is an attempt to more deeply look at this assertion. Governments have increasingly been focusing on moving to a carbon-neutral economy[source, make more specific], societal demands for more sustainability in many aspects of life and progress have been growing, and considerations outside of raw returns in the investment community seem to become more pronounced. Large investors such as pension funds are reacting to individuals demands of higher ethical standards of their investments, e.g. by banning unethical companies from their portfolios \footnote{\href{https://www.dr.dk/nyheder/indland/pensionskasser-vil-afvise-selskaber-med-daarlig-skattemoral}{Danish Pension Funds} are faced with demands from their investors to allocate capital in more sustainable and ethical ways. A closer look on the individual investment vehicle is required by pension members.}. As prices are a function of demand and supply, such a large demand outflow could severely affect prices.

In this paper we consider how returns have been affected by their ESG scores. First, we apply a number multi-factor models to risk adjust portfolios based different degrees of sustainability and find that the ESG levels are not generally priced, but they become priced in times when investors are more flush with cash. We distinguish between good and bad times and observe that ESG firms seem to outperform the market in good times. Interestingly, low ESG firms yield higher abnormal returns than high ESG firms, suggesting that investors forgo return by investing more sustainable. Additionally, we want to consider how higher order moments differ relating to their ESG. As well as if holder composition differs amongst these ESG portfolios. 

Secondly, using an event-study approach as in \citet{Campbell1997} we see the effects of an ESG improvement or depreciation of the firms on their equity returns. We find that a drop in ESG  has a significant effect, but an improvement does not.\footnote{Swap so that is other way round?}

%Investors consider the ethical conditions of companies they invest in, and we show that they do well, before they do good. [Only in good times, investors in stocks are willing to accept a lower expected return as compensation for a higher ethical score]. As we see that the intercept in a regression of the return of long high-ESG short low-ESG portfolio onto risk factors is significantly negative in good times. Good times are defined as [a price-earnings ratio (PE) higher than the rolling ten year average]. Climate Sentiment, a measure of investors interest in climate issues, partially explains this fact, as Climate Sentiment tends to be high in good times. In this analysis Climate Sentiment [is constructed using the relative Google search volume by the term Sustainability.]

\hl{[Write why these methods are useful for this setup/problem]
	
	The results are robust to...
}

These results are interesting as they shed new light on how CEO's should think about their ESG policies, as we now live in a world where investors are becoming more ESG aware. The results suggest, that they should only improve their ESG scores, if they can sustain the higher score. And that they should improve their scores in good times. It is also interesting for fund managers, as there seems to be a significant alpha to investing in firms that have experienced an ESG downgrade.

The remainder of this paper is structured as follows. A literature review sheds light on existing studies on sustainable and ethical investments. The section thereafter elaborates on the data and methodology applied in the analysis. The results report our findings. Finally, we conclude the paper and suggest future venues of research within this field.


\subsection*{Related Literature}

NEW: Optimal time-consistent macroprudential policy JPE 2018

https://libertystreeteconomics.newyorkfed.org/2018/10/regulatory-changes-and-the-cost-of-capital-for-banks.html

https://www.newyorkfed.org/research/staff_reports/sr854


\cite{MiltonFriedman1970} once famously argued that the only firm’s social responsibility is to maximize its owners profits. Any activity that does not pursue this very objective is not considered worthy by investors. But what if an increase in social responsibility and sustainability goes hand in hand with value maximization? This paper investigates the relationship between sustainability and financial performance. 

[Other studies that show ethics has no effect on stock valuation] [Studies that show that it does] [Other studies that look into climate finance] [Other studies that look into sentiment]

The paper is related to two strands of literature: Climate Finance and Sentiment. Climate Finance, and specifically sustainable investments has gained momentum over the past one or two decades. Many claim that it is not only return they are after, but increasingly also want to incorporate other aspects in their investment decisions as, for instance, how social and sustainable firms are. They claim to consider social norms to similar extents as performance. Considering this holds true across different markets and investors, we should observe performance differences according to the degree of sustainability of firms. We define the measure of sustainability through the concept of ESG, a still relatively new concept to assign monetary scores to companies according to their consideration of environmental, social, and governmental factors. The concept of ESG scores has become so popular amongst the investment community that fund managers have long started ESG-only mutual or exchange-traded funds.\footnote{Asset managers like \href{https://www.ishares.com/uk/individual/en/themes/sustainable-investing?switchLocale=y&siteEntryPassthrough=true}{BlackRock}, \href{https://www.jpmorgan.com/country/DK/en/detail/1320566638713}{JPMorgan}, \href{https://www.ubs.com/global/en/asset-management/investment-capabilities/sustainability.html}{UBS}, and many others have long started their own ESG funds, in which they pick particular companies that fulfills their investment agenda. They often refer to them as sustainable and impact investing.} 

Other studies have looked at similar objectives with ambiguous results. Some find a positive relationship between ESG performance and financial performance, whereas other find the opposite or entirely reject the existence of a significant causality. Differences in findings partly root to an unclear definition sustainability. Some of these studies have been conducted on the very same concept of ESG scores, whereas others used the framework of corporate social responsibility (CSR) to link corporates' behavior to performance.

More recent studies, in particular, argue that an increased commitment on social and environmental issues goes hand in hand with positive impacts on firm value and performance. For example, \citet{Dimson2015} review U.S. public companies that engage in environmental, social, and governance concerns and find that commitments come along with positive abnormal returns. Additionally, unsuccessful commitments do not generate significant abnormal returns in either direction. They also present findings showing that successful engagements lead to better accounting performance and increased institutional ownership, suggesting that, in particular, institutions keep a close eye on their investments wider concerns. \citet{Eccles2014} exhibit similar results. They review a sample of 180 firms in the U.S. show that more sustainable firms are more long-term oriented and transparent than their peers. Additionally, they show that high sustainability firms outperform their peers in the long-run. Another study by \citet{Kruger2015} provides evidence on investor’s reactions to events affecting firms’ corporate social responsibility. He finds that negative events triggers a strong negative investor’s reaction and that this reaction is particularly pronounced when these events concern communities and the environment. Other authors, as for instance \citet{Ge2015}, \citet{Fatemi2015}, or \citet{Porter2006} provide additional evidence on positive a relationship between ESG performance and financial performance. Apart from financial performance, literature also suggest out that non-financial performance increases when firms focus commit to more social and environmental sustainability. \citet{Porter1995, Greening2000, Xie2014} refer to, for example, for improved resource productivity, motivated employees, or more customer satisfaction \citep[as cited in][]{Fatemi2018}.
 


Other authors claim that there is no significant relationship between social and environmental sustainability and financial performance. \citet{Alexander1978}, for example, study U.S. firms over the time horizon of 1970 until 1970 and find a low insignificant relationship between social commitment and risk-adjusted returns. Similar findings are presented by \citet{Siegel2000}, arguing that corporate social responsibility has a neutral impact on firm performance and abnormal returns are not significant once returns are risk-adjusted. There is also evidence of no significant relationship provided on a fund-level by, for instance, \citet{Renneboog2008}, \citet{Bauer2005}, or \citet{Hamilton1993}. 

Finally, \citet{Fisher-Vanden2011} and \citet{Boyle1997} both find that social commitment is inversely correlated with financial performance and causes negative abnormal returns. \citet{Fisher-Vanden2011} hypothesizes that it is rather those firms with weak governance standards that engage in environmental initiatives and thereby signal their lack of organizational structure. \citet{Boyle1997}, on the other hand suggest that sustainable engagements will lead to lower cash flows in the future and thus decrease firm value. Evidence on a negative relationship on a fund level is provided \citet{ElGhoul2017} on a fund level, who find that high corporate social responsibility funds perform worse than low funds. They consequently argue that investors thereby must derive non-performance related features out of their engagements when choosing high socially responsibility funds. Coming from another perspective, \citet{Hong2009} have studied 'bad' firms whose business objective revolves around producing and selling alcohol or tobacco, or are involved in gambling. They find that, because investors actively neglect these stocks due to social norms, they have higher expected returns than their peers. Their conclusion is that social norms indeed affect stock prices and returns and investors, as \citet{Hong2009} put it, pay for their discriminatory tastes.

As the literature review suggest, a clear connection between sustainability and financial performance remains yet to be proven with more certainty. We add to the literature by investigating the link between the commitment of social responsibility to firms’ performance from a new perspective. First, we merge the CRSP monthly equity returns with ESG scores from Thomson Reuters, which serve as an indicator for firms’ social commitment. This leaves us with an extensive database that, to our knowledge, has not yet been investigated by any other study. We then form decile portfolios based on previous year ESG scores and apply risk-adjusted factor models to examine financial performance with regards to abnormal returns. Furthermore, we condition the factor models on a bad times indicator and empirically examine changes in responsibility ratings of firms in the spirit of an event study. 

%Write more on concept of ESG scores.

\section{Data and Methodology}

This section outlines how we use relevant data to empirically answer the relevant research questions of this paper. It is outlined as followed. First, we describe the data sample. Second, we go into detail on how to construct ESG based portfolios. Third, multifactor-models to risk-adjust for returns are reviewed. Finally, we are interested in whether volatility in ESG scores are causally connected to returns in equities and we develop a measure to test against this hypothesis.

\subsubsection*{Returns}

The objective of the analysis requires us to combine data on equity returns and sustainability. First, we obtain monthly stock returns from the Center for Research in Security Prices (CRSP). We also obatin monthly data points on the number of stocks and according share price to compute company values.

\subsubsection*{ESG}
We download yearly ESG score data from Thomson Reuters, an equal-weighted rating on companies' sustainability focuses with regards to economic, environmental, social and corporate governance pillars (referred to as the \textit{ASSET4's} pillar). In particular, the ESG score is a measure from 0 to 100. A low score suggests that a given company behaves poorly with regards to sustainability, and vice versa. The higher the company score, the more sustainable it is with regards to the three pillars.\footnote{The interested readers can find a detailed description on how Thomson Retuers determines their ESG scores \href{http://www.esade.edu/itemsweb/biblioteca/bbdd/inbbdd/archivos/Thomson_Reuters_ESG_Scores.pdf}{here}.} We merge the return data from CRSP with the ESG data according to their CUSIP codes. As they differ in time instances, ESG data points are the same throughout the every month of a single year and are updated only once after each year. ESG scores are available from 2002 until today (2017), which therefore defines our sample period.

\subsubsection*{Risk Factors}
To control for risk factors we use the risk-free rate and factor-returns of Fama-French's five factors as well as the momentum factor from \href{https://mba.tuck.dartmouth.edu/pages/faculty/ken.french/data_library.html}{Ken French's website}.

\subsubsection*{[Good times]}
We define good times to be periods where the current price-earnings ratio (PE) is above Shiller's cyclically adjusted price-earnings ratio (CAPE). Shiller calculates the CAPE as the 10-year rolling average PE of what is now called the S\& P 500, on a monthly basis using end-of-day end-of-month prices and earnings are linearly interpolated from the quarter earnings. Alternative specifications we also test, is defining good times more broadly as times with high PE ratios. Data is gathered from \href{http://www.econ.yale.edu/~shiller/data.htm}{Shiller's website}.

\subsubsection*{[Sentiment]}
We test for sentiment by using the search interest of the word 'Sustainability' on Google. The measure is relative search volume in the United States. A breakdown into states is also possible. The data is retrieved from \href{}{Google Trends}.

[Use Wurgler Investor Sentiment measure instead?]

\subsection{Summary Statistics}

We begin by investigating the ESG data set in greater detail. Table~\ref{tab:descriptive} exhibits distribution statistics and developments in ESG scores over time. In the first year of the sample period, 2002, a total number of 573 firms in the sample list an ESG score. This number significantly increases to 2,701 firms in the final year of the sample, 2016. The distribution of ESG scores remains relatively stable over time with a mean score in between approximately 40 to 60. Scores on the very low end as well as on the high end are found.

Figure~\ref{fig:esg_distribution} in Appendix~\ref{app:esgscores} plots ESG scores over all scores available and across companies' yearly averages. It is striking many scores are found in the upper and lower score distribution. This suggests that a company would rather exhibit a low score than not having one at all despite the fact that a low score implies low sustainability.

\begin{table}[!htbp] \centering 
	\caption{ESG descriptive data}
	\fnote{The table covers the descriptive statistics of the ESG data set used in the analysis. The minimum, quartiles, maximum and standard deviations are calculated (equally-weighted) over all companies exhibiting an ESG score for a given year. It ranges from 573 companies that excibit an ESG score in 2002 up to 2,701 in 2016. The distribution estimates remain relatively stable across the years except for the last two years of 2015 and 2016.}
	\label{tab:descriptive} 
	\begin{tabular*}{\textwidth}{@{\extracolsep{5pt}} ccccccccc} 
		\\[-1.8ex]\hline 
		\hline \\[-1.8ex] 
		& \# of firms & Min & 1. Quartile & Median & Mean & 3. Quartile & Max & Std \\ 
		\hline \\[-1.8ex] 
		2002 & $573$ & $3.26$ & $21.15$ & $42.29$ & $48.78$ & $79$ & $98.72$ & $30.73$ \\ 
		2003 & $583$ & $3.80$ & $20.54$ & $42.95$ & $48.89$ & $78.60$ & $98.68$ & $30.54$ \\ 
		2004 & $805$ & $3.74$ & $31.21$ & $55.46$ & $56.33$ & $84.40$ & $98.38$ & $28.17$ \\ 
		2005 & $924$ & $5.44$ & $33.01$ & $56.55$ & $58.05$ & $86.25$ & $98.49$ & $28.36$ \\ 
		2006 & $917$ & $4.25$ & $32.82$ & $56.59$ & $57.81$ & $86.71$ & $98.25$ & $28.45$ \\ 
		2007 & $948$ & $3.88$ & $31.54$ & $59.73$ & $58.31$ & $86.72$ & $97.30$ & $28.38$ \\ 
		2008 & $1,150$ & $4.32$ & $26.96$ & $51.69$ & $54.53$ & $85.76$ & $97.50$ & $29.55$ \\ 
		2009 & $1,287$ & $3.54$ & $27.23$ & $49.84$ & $54$ & $84.89$ & $97.46$ & $29.69$ \\ 
		2010 & $1,321$ & $4.40$ & $30.22$ & $54.80$ & $57.01$ & $86.90$ & $97.10$ & $28.63$ \\ 
		2011 & $1,310$ & $4.73$ & $28.82$ & $58.55$ & $57.24$ & $86.98$ & $96.60$ & $29.01$ \\ 
		2012 & $1,310$ & $4.28$ & $26.95$ & $54.91$ & $55.31$ & $85.88$ & $96.80$ & $29.51$ \\ 
		2013 & $1,326$ & $4.50$ & $28.76$ & $55.39$ & $56.44$ & $86.74$ & $96.95$ & $29.32$ \\ 
		2014 & $1,330$ & $3$ & $31.50$ & $58.69$ & $57.54$ & $86.04$ & $97.11$ & $28.69$ \\ 
		2015 & $1,972$ & $4.59$ & $15.03$ & $44.07$ & $48.10$ & $81.95$ & $96.59$ & $32.26$ \\ 
		2016 & $2,701$ & $4.84$ & $15.47$ & $27.25$ & $43.34$ & $78.37$ & $96.43$ & $31.97$ \\ 
		\hline \\[-1.8ex] 
	\end{tabular*} 
\end{table} 

Out of 57 firms that were part of the highest decile ESG scores in 2002, a significant number of 33 were also part of this portfolio in the end of the sample, suggesting that ESG scores are sticky in the top decile, see Table~\ref{tab:high_esg_companies} in Appendix~\ref{app:esgscores}. Interestingly, also firms that one would think are not part of that group, as for example British American Tobacco PLC or Occidental Petroleum Corporation, are members of the high profile ESG group. This suggests that not the objective of the firm matters but instead how well the criteria to obtain a high score are fulfilled. Though this procedure seems rather arbitrary, it proves to allow every firm to obtain a high score regardless of their business model.

For the empirical analysis in the next chapter, only ESG score firms are taken into account. The total number of firms in every is thereby identical to the number of firms in Table~\ref{tab:descriptive}. This also implies that the cross-section's total number of firms in every portfolio rises significantly over time. Whereas there are only 57 firms in each decile portfolio in 2003, there is a total number of 270 firms in each decile portfolio in the year of 2017.

\subsection{Portfolio Construction}

We subdivide into ESG disclosure and non-disclosure companies. Within the ESG-score disclosure companies, we distinguish between high and low scores. In total, we subdivide our sample into ten portfolios, ranging from the highest to the lowest decile ESG companies. According to these portfolios, we rearrange our returns for the following year. For example, ESG scores in 2002 determine our portfolios in 2003 and so forth. We construct  equally-weighted decile portfolios for the entire data period, where P1 (P10) depicts the highest (lowest) ESG portfolio. We choose to equally-weight, because the data sample is small in the first few year of the sample.\footnote{If value-weighted, big firms pick up more than a quarter of the total return in some of the portfolios at some points in time. It would therefore be almost entirely driven by big firms and not properly represent the entire universe of the ESG portfolios.} However, one should note that that value composition between the portfolios is not evenly distributed, see Figure~\ref{fig:sizedistr} in Appendix~\ref{app:ESGportfolios}. It seems that high scores are primarily obtained by rather large firms, and vice versa. Finally, ee use the self-developed portfolios to construct a long-short portfolio (LS), which goes goes long in the highest ESG decile portfolio and shorts the lowest ESG decile portfolio. 



\subsection{Risk-adjusting Returns}

A bold comparison of cumulative returns exhibits first insights in how ESG scores impact return profiles. Secondly, we run the the capital asset pricing model (CAPM), Fama-French three-factor, the Carhart four-factor, and the Fama-French five factor model on the portfolios to risk-adjust returns \citep[see][]{Sharpe1964,Fama1992,Carhart1997,Fama2015}. We thereby follow the regression approach of


\begin{equation}
\label{eq:riskadjustment}
r_{it} - r_t^f = \alpha_i + \sum_{j=1}^{n} \beta_{ij} f_{jt} + \epsilon_{it},
\end{equation}
%CHECK FORMULA!

where $r_{it}$ depicts portfolio's $i$'s return at time $t$. Moreover, $r_t^f$, $\alpha_i$, and $n$ denote the risk-free rate, the abnormal return, and the number of factors. Finally, the $\beta_{j}$, $f_{jt}$ and $\epsilon_{it}$ are the factors, factor loadings and the error term. We run the regressions on both equally-weighted portfolios.

Secondly, we are interested in whether high and low ESG portfolios react different under varying economic environments. We introduce the condition of good and bad times. We split $\alpha_i$ from equation~\eqref{eq:riskadjustment} into two abnormal return coefficients \{$\alpha^0$, $\alpha^B$\} by introducing a binary variable that indicates whether we are in good or bad times. We compute 

\begin{equation}
\label{eq:goodandbadtimes}
r_{it} - r_t^f = 
	\underbrace{\alpha_i^{G} G_t}_{\substack{Abnormal\\Return~in\\Good~Times}} + 
	\underbrace{\alpha_i^{B} B_t}_{\substack{Abnormal\\Return~in\\Bad~Times}}  
	+ \sum_{j=1}^{n} \beta_{ij} f_{jt} + \epsilon_{it},
\end{equation}

where $G_t$ equals 1 in good times and 0 otherwise. $B_t$, on the other hand, equals 1 in bad times and 0 otherwise. This means that the factor loadings of $\alpha_i^{G}$ and $\alpha_i^{B}$ capture the abnormal performance for every decile portfolio $i$ according to good and bad times. The conditional factor models imply that $\alpha^G = 0$ and $\alpha^B = 0$ \citep[see, for example,][for similar applications]{Ferson2009, Christopherson1998}.


\subsection{Volatility in ESG Scores}

Finally, we are interested if there is a causality between fluctuations in ESG scores and equity returns. The hypothesis is that the ESG score is seen by investors as an additional source of risk and that volatility in that score is punished with lower cumulative abnormal returns. We follow an approach by \citet{Campbell1997}.

First, we calculate changes in the ESG scores from one year to another, meaning that we define $\Delta ESG = ESG_{t-2} - ESG_{t-1}$, where $t$ is measured in years. We assume that investors do not know of the ESG score of the current year until the very last day of this particular year. If $\Delta ESG$ is larger (smaller) than the threshold of $X \in \{ -40, -35, ..., 35, 40\}$, it will be be part of the event study's sample. We assume that investors do not have knowledge of the ESG score in the current year but only when the year ends, so  we are interested in the returns after the change occurred. For example, assume that firm A has an ESG score of 80 in year 2010 and of 45 in year 2011. We assume that the score of 45 for year of 2011 was made public in December of 2011 and we are then then interested in the returns in the beginning of 2012 as investors do not know the ESG score of 2011 until the end of the year.

We apply the CAPM model to estimate expected returns based on the factor loading 18 months prior to the new year. In prior the example this relates to an estimation period of July 2010 until December 2011. Our time line looks as follows.



\begin{figure}[!htpb]
	\centering
	\begin{tikzpicture}
	
	\usetikzlibrary{arrows,decorations.pathreplacing}
	
	\tikzset{number line/.style={}}
	
	\tikzset{
		brace_top/.style={
			color=black,
			decoration={brace},
			decorate
		},
		brace_bottom/.style={
			color=black,
			decoration={brace, mirror},
			decorate
		}
	}
	
	\draw (0,0) -- (15,0);
	\foreach \x in {0.8, 3, 7.5, 8.5, 10.5, 14.2}
	\draw(\x cm,3pt) -- (\x cm, -6pt);
	\draw (0.8,0) node[above=3pt] {$T_0 = -18$};
	\draw (3,0) node[above=3pt] {$T = -13$};
	\draw (7.5,0) node[above=3pt] {$T_1 = -1$};
	\draw (8.5,0) node[above=3pt] {$0$};
	\draw (10.5,0) node[above=3pt] {$T_2 = 5$};
	\draw (14.2,0) node[above=3pt] {$T_3 = 12$};
	\draw (4,0) node[above=18pt, align=center] {
		$\left(\mytab{estimation window}\right]$};
	\draw (9,0) node[above=18pt, align=center] {
		$\left(\mytab{event window}\right]$};
	\draw (12.3,0) node[above=18pt, align=center] {
		$\left(\mytab{post-event window}\right]$};
	
	\node (3,-0.5) at (3,-0.5) {};
	\node (7.5,-0.5) at (7.5,-0.5) {};
	\draw [brace_bottom] (3,-0.5) -- node [below=3pt, pos=0.5] {\mytab{$\Delta$ESG \\ above threshold}} (7.5,-0.5);
	
	\end{tikzpicture}
\end{figure}


We use the calculated expected returns and compare them to actually realized returns. We then compute the differences, measured by abnormal returns, and cumulate them over a time horizon of 5 months. As in in \citet{Campbell1997}, we assume the market model to hold true, meaning that for any security $i$, we expect a return of 

\begin{equation}
R_{it} = \alpha_i + \beta_i  R_{mt} + \epsilon_{it},
\end{equation}

where $R_{it}$ and $R_{mt}$ are the excess returns on firm $i$ and the markets excess return, both at time $t$. Furthermore, $E[\epsilon_{it}] = 0$ and $Var[\epsilon_{it}] = \sigma^2_{\epsilon_{i}}$. The difference between the expected and the actual return then depicts the abnormal return for a given month. We cumulate abnormal returns over a time horizon of 5 months and derive the relevant test statistics of $J_1$ and $J_2$.

%Finally, we vary the event window to check for robustness.



\section{Results}

In this section, we empirically investigate the relationship between ESG scores and equity returns. The chapters exhibit results of computed portfolios in good and bad times. Furthermore, we adjust for risk through multi-factor models, and investigate volatility in ESG scores and its implications on returns.

\subsection{Empirical Analysis}

The empirical analysis consists of two major components. First, we risk-adjust returns of our ESG decile portfolios through the application of factor-models. We further condition on good and bad in the economy and thereby split abnormal returns in different economic environments. We then attempt to test return differences in high and low ESG portfolios against other explanatory variables. 

Secondly, we review changes in ESG scores and what they mean for the performance of firms. Specifically, we apply an event-study approach to calculate cumulative abnormal returns after changes in scores.


\subsubsection{Risk-adjusting returns}

We construct 10 equally-weighted decile portfolios and risk-adjust through four risk-factor models to test for abnormal returns. The results are displayed in Table~\ref{tab:riskadjustments}. We report excess returns, alphas under the CAPM, 3-factor, 4-factor, 5-factor model as well as monthly volatility and Sharpe ratio estimates.

The findings mostly suggest no significant relationship between sustainability as measured in ESG and returns. However, there is some evidence that portfolios to the extremes (high and low ESG firms) tend to outperform others. Also, the long-short portfolio (LS) exhibits a negative significant return under the 4- and 5-factor model, implying that it is unprofitable to invest (go long) in high ESG firms and short low ESG firms. Instead, the opposite would have earned positive abnormal returns in the investigated time horizon. Doing so would have yielded a significant abnormal monthly return of 0.244\% under the 4-factor model and 0.309\% under the 5-factor model. 

\begin{table}[!htbp] \centering 
	\caption{Risk-adjusted US Equity Returns}
	\fnote{We construct equally-weighted decile portfolios based on previous year ESG scores and adjust them in the beginning of each calender year. P1 (P10) depicts the high (low) ESG score portfolio. LS is a time series of returns that goes long P1 and shorts P10. The returns of the equally-weighted ESG portfolios are risk-adjusted through the application of the CAPM, 3-factor, 4-factor, and 5-factor models. We further disclose monthly excess returns, volatility and Sharpe ratio estimates. $t-values$ test if the estimated returns are significantly different from zero and bold numbers signal significance to the 10\% level or less.} 
	\label{tab:riskadjustments} 
	\resizebox{\textwidth}{!}{\begin{tabular}{@{\extracolsep{5pt}} lccccccccccc} 
			\\[-1.8ex]\hline 
			\hline \\[-1.8ex] 
			& P1 & P2 & P3 & P4 & P5 & P6 & P7 & P8 & P9 & P10 & LS \\ 
			\hline \\[-1.8ex] 
			Excess Return & \textbf{1.004} & \textbf{1.126} & \textbf{ 1.105} & \textbf{1.102} & \textbf{1.014} & \textbf{0.947} & \textbf{0.938} & \textbf{1.166} & \textbf{1.128} & \textbf{1.318} & \textbf{-0.410} \\ 
			t-value & 2.991 & 3.419 & 2.994 & 2.763 & 2.649 & 2.469 & 2.473 & 2.778 & 2.709 & 3.021 & -2.313 \\[2.5ex] 
			
			CAPM alpha & 0.103 & \textbf{0.256} & 0.140 & 0.049 & 0.003 & -0.060 & -0.060 & 0.085 & 0.056 & 0.203 & -0.198 \\ 
			t-value & 1.192 & 2.489 & 1.111 & 0.391 & 0.027 & -0.480 & -0.495 & 0.533 & 0.354 & 1.172 & -1.211 \\[2.5ex] 
			
			3-factor alpha & 0.108 & \textbf{0.263} & 0.154 & 0.054 & 0.010 & -0.057 & -0.055 & 0.095 & 0.062 & 0.210 & -0.202 \\ 
			t-value & 1.247 & 2.563 & 1.267 & 0.466 & 0.093 & -0.512 & -0.526 & 0.707 & 0.450 & 1.355 & -1.431 \\[2.5ex]  
			
			4-factor alpha & 0.126 & \textbf{0.289} & \textbf{0.200 }& 0.097 & 0.053 & -0.014 & -0.016 & 0.156 & 0.110 & \textbf{0.272} & \textbf{-0.244} \\ 
			t-value & 1.517 & 2.965 & 1.895 & 0.951 & 0.584 & -0.145 & -0.177 & 1.421 & 0.895 & 2.029 & -1.869 \\[2.5ex]  
			
			5-factor alpha & 0.126 & \textbf{0.247} & 0.132 & 0.073 & 0.043 & -0.066 & -0.044 & 0.123 & 0.131 &\textbf{0.335} & \textbf{-0.309} \\ 
			t-value & 1.410 & 2.320 & 1.051 & 0.602 & 0.389 & -0.569 & -0.415 & 0.882 & 0.933 & 2.138 & -2.170 \\[2.5ex] 
			
			Volatility & 4.502 & 4.416 & 4.948 & 5.347 & 5.129 & 5.140 & 5.079 & 5.624 & 5.582 & 5.845 & 2.386 \\ 
			Sharpe Ratio & 0.223 & 0.255 & 0.223 & 0.206 & 0.198 & 0.184 & 0.185 & 0.207 & 0.202 & 0.225 & -0.172 \\ 
			\hline \\[-1.8ex] 
	\end{tabular}}
\end{table} 

As mentioned, it seems as if portfolios to the extremes tend to outperform others. Though not significant in most cases, alphas  across decile portfolios are U-shaped. 

Volatility rises almost monotonically from the lowest to the highest ESG portfolio, suggesting that low ESG firms are more risky. An explanation could root in the value composition of portfolios. As shown in Figure~\ref{fig:sizedistr} in Appendix~\ref{app:ESGportfolios} the total market value of high-ESG portfolios is significantly larger than low-ESG portfolios, suggesting that it is rather large firms having high scores. As large firms are typically less volatile than small firms, the increase from high- to low-ESG portfolios is reasonable.

Sharpe ratios are also U-Shaped; comparing volatility estimates with Sharpe ratios implies that expected returns for low ESG firms are higher. We know that, on average, small firms have higher expected returns than large firms. As just mentioned, it is rather small firms that belong to low-ESG portfolios and therefore it is plausible that these portfolios also exhibit higher expected returns. 



\subsubsection{Risk-adjusting returns in good and bad times}

[Make table here of the ESG return in good times. PE ratio main specification, then good times alpha. For five factor? or Carhart?]

Secondly, we risk-adjust returns conditionally on good and bad times as expressed in equation~\eqref{eq:goodandbadtimes}, where $\alpha^G$ and $\alpha^B$ exhibit abnormal returns in good (G) and bad (B) times. Go od times (bad times) are defined by the current P/E ratio being above (below) the current price over the 10-year rolling average earnings. Table~\ref{tab:goodandbad} shows the results. 


\begin{table}[!htbp] \centering 
	\caption{Equally-Weighted with badCAPEtimes Dummy} 
	\fnote{The returns of the equally-weighted portfolios are risk adjusted through the application of the CAPM, 3-factor, 4-factor, and 5-factor models. The LS portfolio goes long in the highest ESG decile firms and shorts the lowest ESG decile firms. Additionally, we condition the regression on good and bad times through binary variables as in $r_{it} - r_t^f = \alpha_i^G G_t + \alpha_i^B B_t + + \sum_{j=1}^{n} \beta_{ij} f_{jt} + \epsilon_{it}$. Here, $\alpha^G$ exhibits the abnormal return in good times, whereas $\alpha^B$ depicts the abnormal return that is additionally earned in bad times. Good times (bad times) are defined by the current P/E ratio being above (below) the current price over the 10-year rolling average earnings. The tables depicts alphas in percentage points. $t-values$ test if the estimated abnormal returns are significantly different from zero and bold numbers signal significance to the 10\% level or less.}
	\label{tab:goodandbad} 
	\resizebox{\textwidth}{!}{\begin{tabular}{@{\extracolsep{5pt}} lccccccccccc}
			\\[-1.8ex]\hline 
			\hline \\[-1.8ex] 
			V1 & P1 & P2 & P3 & P4 & P5 & P6 & P7 & P8 & P9 & P10 & LS \\ 
			\hline \\[-1.8ex] 
			CAPM alpha$^G$ & \textbf{0.513} & \textbf{0.728} & \textbf{1.068} & \textbf{0.766} & \textbf{0.888} & \textbf{0.891} & \textbf{0.578} & \textbf{1.179} & \textbf{0.914} & \textbf{2.054} & \textbf{-1.541} \\ 
			t-value & 2.278 & 2.709 & 3.292 & 2.37 & 2.893 & 2.762 & 1.819 & 2.876 & 2.214 & 4.763 & -3.7 \\ [2ex] 
			
			CAPM alpha$^B$ & 0.033 & 0.176 & -0.018 & -0.074 & -0.148 & \textbf{-0.223} & -0.17 & -0.102 & -0.091 & -0.114 & 0.147 \\ 
			t-value & 0.353 & 1.585 & -0.138 & -0.556 & -1.173 & -1.679 & -1.297 & -0.606 & -0.533 & -0.641 & 0.854 \\[4ex] 
			
			
			3-factor alpha$^G$ & \textbf{0.596} & \textbf{0.765} & \textbf{1.039} & \textbf{0.587} & \textbf{0.675} & \textbf{0.634} & \textbf{0.302} & \textbf{0.841} & \textbf{0.554} & \textbf{1.755} & \textbf{-1.159} \\ 
			t-value & 2.626 & 2.814 & 3.283 & 1.903 & 2.413 & 2.154 & 1.082 & 2.367 & 1.517 & 4.435 & -3.151 \\ [2ex] 
			
			3-factor alpha$^B$ & 0.025 & 0.178 & 0.004 & -0.036 & -0.103 & -0.175 & -0.116 & -0.031 & -0.022 & -0.052 & 0.076 \\ 
			t-value & 0.267 & 1.613 & 0.029 & -0.288 & -0.905 & -1.459 & -1.02 & -0.216 & -0.145 & -0.32 & 0.51 \\[4ex] 
			
			
			4-factor alpha$^G$ & 0.375 & 0.436 & 0.446 & 0.001 & 0.105 & 0.066 & -0.24 & 0.019 & -0.11 & \textbf{1.019} & \textbf{-0.644} \\ 
			t-value & 1.622 & 1.603 & 1.517 & 0.004 & 0.414 & 0.242 & -0.934 & 0.063 & -0.32 & 2.763 & -1.778 \\[2ex] 
			
			4-factor alpha$^B$ & 0.082 & 0.263 & 0.156 & 0.115 & 0.044 & -0.028 & 0.024 & 0.18 & 0.149 & 0.138 & -0.056 \\ 
			t-value & 0.891 & 2.44 & 1.34 & 1.013 & 0.435 & -0.264 & 0.233 & 1.487 & 1.1 & 0.945 & -0.393 \\[4ex] 
			
			
			5-factor alpha$^G$ & \textbf{0.647} & \textbf{0.777} & \textbf{1.07} & \textbf{0.597} & \textbf{0.76} & \textbf{0.694} & 0.43 & \textbf{0.936} & \textbf{0.67} & \textbf{1.955} & \textbf{-1.308} \\ 
			t-value & 2.806 & 2.808 & 3.335 & 1.9 & 2.688 & 2.327 & 1.553 & 2.601 & 1.822 & 5.02 & -3.584 \\[2ex] 
			
			5-factor alpha$^B$ & 0.038 & 0.158 & -0.026 & -0.016 & -0.078 & -0.194 & -0.124 & -0.014 & 0.041 & 0.062 & -0.024 \\ 
			t-value & 0.399 & 1.385 & -0.197 & -0.121 & -0.671 & -1.578 & -1.086 & -0.096 & 0.268 & 0.385 & -0.159 \\ 
			\hline \\[-1.8ex] 
	\end{tabular} }
\end{table} 


We find that times indeed matter for the risk-adjusted performance of ESG-sorted portfolios. In good times, risk-adjusted abnormal returns are significant in good times under most of the factor models. Also, low ESG firms significantly outperform high ESG firms in good times. The long-short (LS) portfolio's excess return is significantly different from zero under all specifications. This means that, in good times, investors are better off to go long in low ESG stocks. When intending to minimize market exposure and also going short in high ESG firms, investors earn a significant abnormal return through such a trading strategy.

These results are new to the literature and surprising. It suggests that, in good times, investors willingly forgoe abnormal return by neglecting low ESG stocks. One could argue that investors receive non-monetary utility in these times from instead investing in high ESG firms. 

We further document the regression results of our long-short ESG portfolio (LS) against all factor models in Table~\ref{tab:goodandbadfactor}. Interestingly, the LS portfolio is not market neutral and is further exposed to the small-minus-big (SMB) factor to a significant level. Furthermore, the momentum factor picks up a lot of the explanatory power. As already shown in Table~\ref{tab:goodandbad}, the return on the LS portfolio is significant in good times across all factor models.



\begin{table}[!htbp] \centering 
	\caption{Equally-Weighted long-short ESG portfolio in good and bad times (badCAPEtimes Dummy)} 
	\fnote{The table exhibits the regression results of the long-short (LS) ESG portfolio (LS) against the CAPM, 3-factor, 4-factor, and 5-factor models. The LS portfolio goes long in the highest ESG decile firms and shorts the lowest ESG decile firms. The regressions condition on good and bad times in the economy as in $LS_{t} = \alpha^G G_t + \alpha^B B_t + \sum_{j=1}^{n} \beta_{j} f_{jt} + \epsilon_{t},$ where $\alpha^G$ exhibits the abnormal return in good times, and $\alpha^B$ depicts the abnormal return that is earned in bad times. Good times (bad times) are defined by the current P/E ratio being above (below) the current price over the 10-year rolling average earnings. The tables depicts alphas in percentage points. Standard errors are in the brackets below the coefficients.}
	\label{tab:goodandbadfactor} 
	\begin{tabular*}{\textwidth}{l @{\extracolsep{\fill}} cccc}
		\\[-1.8ex]\hline 
		\hline \\[-1.8ex] 
		& \multicolumn{4}{c}{\textit{Dependent variable:}} \\ 
		\cline{2-5} 
		\\[-1.8ex] & \multicolumn{4}{c}{LS} \\ 
		\\[-1.8ex] & (1) & (2) & (3) & (4)\\ 
		\hline \\[-1.8ex] 
		alpha$^G$ & $-$1.541$^{***}$ & $-$1.159$^{***}$ & $-$0.644$^{*}$ & $-$1.308$^{***}$ \\ 
		& (0.416) & (0.368) & (0.362) & (0.365) \\ 
		& & & & \\ 
		alpha$^B$ & 0.147 & 0.076 & $-$0.056 & $-$0.024 \\ 
		& (0.172) & (0.150) & (0.143) & (0.150) \\ 
		& & & & \\ 
		Market & $-$0.277$^{***}$ & $-$0.140$^{***}$ & $-$0.078$^{**}$ & $-$0.089$^{**}$ \\ 
		& (0.040) & (0.039) & (0.039) & (0.041) \\ 
		& & & & \\ 
		SMB &  & $-$0.455$^{***}$ & $-$0.490$^{***}$ & $-$0.411$^{***}$ \\ 
		&  & (0.065) & (0.062) & (0.066) \\ 
		& & & & \\ 
		HML &  & $-$0.098$^{*}$ & 0.004 & $-$0.148$^{**}$ \\ 
		&  & (0.059) & (0.059) & (0.065) \\ 
		& & & & \\ 
		Mom &  &  & 0.166$^{***}$ &  \\ 
		&  &  & (0.034) &  \\ 
		& & & & \\ 
		RMW &  &  &  & 0.264$^{***}$ \\ 
		&  &  &  & (0.091) \\ 
		& & & & \\ 
		CMA &  &  &  & 0.177 \\ 
		&  &  &  & (0.110) \\ 
		& & & & \\ 
		\hline \\[-1.8ex] 
		Observations & 180 & 180 & 180 & 180 \\ 
		Adjusted R$^{2}$ & 0.248 & 0.431 & 0.496 & 0.456 \\ 
		\hline 
		\hline \\[-1.8ex] 
		\textit{Note:}  & \multicolumn{4}{r}{$^{*}$p$<$0.1; $^{**}$p$<$0.05; $^{***}$p$<$0.01} \\ 
	\end{tabular*} 
\end{table} 

We also run the regressions on a LS portfolio that is goes long the highest 20\% of ESG firms and goes short in the lowest 20\% and find very similar results. The return difference of the LS-portfolio remains significant across all risk-adjustments. Furthermore, we redefine our good and bad times dummy to check the robustness of our good and bad times estimator. We redefine good times through the condition that the current share price over current earnings needs to be above the current share price over the latest 2-year rolling average of earnings instead of over the a 10-year rolling average, and vice versa. Here, too, we find similar results and the negative LS-portfolio returns remain significant in good times.


\subsection{The Long-Short portfolio and other explanatory variables}

We attempt to explain the abnormal return through other variables. First, we regress the LS-portfolio against the levels of the PE ratio itself. The level in CAPE (PE) is defined as the current share price over the 10-year (2-year) rolling average. Table~\ref{tab:PEandCAPEexplanatory} exhibits the results. By default, we choose to always incorporate the four-factor model as this was giving us the highest explanatory power out of all the factor models and where alpha$^G$ was at its lowest. 

In Table~\ref{tab:PEandCAPEexplanatory}, column (1) states the findings from Table~\ref{tab:goodandbadfactor} and \ref{tab:goodandbad}. Column (2) regresses the LS portfolio against the CAPE regardless of good and bad times, whereas column (3) incorporates good and bad times. Here, alpha$^G$ and alpha$^B$ cannot be interpreted as returns anymore, because CAPE, as one of the explanatory variables, is not a return itself. We find a significant negative relationship between the level of CAPE and the LS portfolio in one of the specifications.

Column (4), (5), and (6) repeat the analysis, but instead consider the level of P/E according the rolling average over the last 2 years. The results are similar. A negative and significant relationship between the return difference of the highest and lowest ESG portfolio and the level of P/E. 

The results imply that the higher and therefore the 'the better' times are, the higher the absolute abnormal return on the LS short portfolio, or stated differently, the higher the difference between the high and the low ESG portfolio. The level of CAPE or P/E and therefore the level of economic times is a significant indicator for returns on sustainable firms. 


\begin{table}[!htbp] \centering 
	\caption{Equally-Weighted long-short ESG portfolio against PE and CAPE} 
	\fnote{The table exhibits the regression results of the long-short (LS) ESG portfolio (LS) against the the four-factor model including specifications on good and bad times. The LS portfolio goes long in the highest ESG decile firms and shorts the lowest ESG decile firms. The condition on good and bad times in the economy is specified in $LS_{t} = \alpha^G G_t + \alpha^B B_t + \sum_{j=1}^{n} \beta_{j} f_{jt} + \epsilon_{t},$ where $\alpha^G$ exhibits the abnormal return in good times, and $\alpha^B$ depicts the abnormal return that is earned in bad times. In column 2 (4), good times in $\alpha^G$ are defined by the current P/E ratio being above the current price over the 10-year (2-year) rolling average earnings, and vice versa. CAPE and PE are levels of the ratios itself. The tables depicts alphas in percentage points. Standard errors are in the brackets below the coefficients.}
	\label{tab:PEandCAPEexplanatory} 
	\begin{tabular*}{\textwidth}{l @{\extracolsep{\fill}} cccccc}
		\\[-1.8ex]\hline 
		\hline \\[-1.8ex] 
		& \multicolumn{6}{c}{\textit{Dependent variable:}} \\ 
		\cline{2-7} 
		\\[-1.8ex] & \multicolumn{6}{c}{LS} \\ 
		\\[-1.8ex] & (1) & (2) & (3) & (4) & (5) & (6)\\ 
		\hline \\[-1.8ex] 
		Constant &  & 0.244 &  &  & 0.246 &  \\ 
		&  & (0.240) &  &  & (0.227) &  \\ 
		& & & & & & \\ 
		alpha$^G$ & $-$0.644$^{*}$ &  & $-$0.011 & $-$0.330$^{*}$ &  & 0.092 \\ 
		& (0.362) &  & (0.604) & (0.181) &  & (0.312) \\ 
		& & & & & & \\ 
		alpha$^B$ & $-$0.056 &  & 0.212 & 0.048 &  & 0.292 \\ 
		& (0.143) &  & (0.250) & (0.186) &  & (0.236) \\ 
		& & & & & & \\ 
		CAPE &  & $-$0.015$^{*}$ & $-$0.012 &  &  &  \\ 
		&  & (0.008) & (0.009) &  &  &  \\ 
		& & & & & & \\ 
		PE &  &  &  &  & $-$0.016$^{**}$ & $-$0.014$^{*}$ \\ 
		&  &  &  &  & (0.008) & (0.008) \\ 
		& & & & & & \\ 
		Mkt.RF & $-$0.078$^{**}$ & $-$0.068$^{*}$ & $-$0.073$^{*}$ & $-$0.073$^{*}$ & $-$0.069$^{*}$ & $-$0.073$^{*}$ \\ 
		& (0.039) & (0.037) & (0.039) & (0.038) & (0.037) & (0.038) \\ 
		& & & & & & \\ 
		SMB & $-$0.490$^{***}$ & $-$0.500$^{***}$ & $-$0.494$^{***}$ & $-$0.500$^{***}$ & $-$0.496$^{***}$ & $-$0.493$^{***}$ \\ 
		& (0.062) & (0.060) & (0.062) & (0.061) & (0.060) & (0.060) \\ 
		& & & & & & \\ 
		HML & 0.004 & 0.005 & 0.002 & 0.018 & 0.001 & 0.003 \\ 
		& (0.059) & (0.058) & (0.059) & (0.058) & (0.058) & (0.058) \\ 
		& & & & & & \\ 
		Mom & 0.166$^{***}$ & 0.154$^{***}$ & 0.153$^{***}$ & 0.177$^{***}$ & 0.150$^{***}$ & 0.151$^{***}$ \\ 
		& (0.034) & (0.035) & (0.036) & (0.033) & (0.036) & (0.036) \\ 
		& & & & & & \\ 
		\hline \\[-1.8ex] 
		Observations & 180 & 180 & 180 & 180 & 180 & 180 \\ 
		Adjusted R$^{2}$ & 0.496 & 0.494 & 0.498 & 0.496 & 0.496 & 0.501 \\ 
		\hline 
		\hline \\[-1.8ex] 
		\textit{Note:}  & \multicolumn{6}{r}{$^{*}$p$<$0.1; $^{**}$p$<$0.05; $^{***}$p$<$0.01} \\ 
	\end{tabular*} 
\end{table} 


We also test the specifications against the other factor models (CAPM, 3-factor, 5-factor). The results are even more pronounced for these specifications. Across all other risk-adjustments, the relationship between CAPE or PE and the return on the LS portfolio is negative and significant, confirming the findings from Table~\ref{tab:PEandCAPEexplanatory}.

%ANDREAS regression tables.
\begin{comment}
% Table created by stargazer v.5.2.2 by Marek Hlavac, Harvard University. E-mail: hlavac at fas.harvard.edu
% Date and time: fr, maj 31, 2019 - 16:20:20
\begin{table}[!htbp] \centering 
  \caption{ESG premium in good times} 
  \label{} 
\small 
\begin{tabular}{@{\extracolsep{5pt}}lcccc} 
\\[-1.8ex]\hline 
\hline \\[-1.8ex] 
 & \multicolumn{4}{c}{\textit{Dependent variable:}} \\ 
\cline{2-5} 
\\[-1.8ex] & \multicolumn{4}{c}{ESG\_deciles[, 11]} \\ 
\\[-1.8ex] & (1) & (2) & (3) & (4)\\ 
\hline \\[-1.8ex] 
 Constant & $-$0.002 &  & 0.005 & 0.005 \\ 
  & t = $-$1.461 &  & t = 2.273$^{**}$ & t = 2.321$^{**}$ \\ 
  & & & & \\ 
 BadCAPETimes &  & $-$0.0002 &  &  \\ 
  &  & t = $-$0.159 &  &  \\ 
  & & & & \\ 
 GoodCAPETimes &  & $-$0.013 &  &  \\ 
  &  & t = $-$3.584$^{***}$ &  &  \\ 
  & & & & \\ 
 CAPE &  &  & $-$0.0003 &  \\ 
  &  &  & t = $-$3.937$^{***}$ &  \\ 
  & & & & \\ 
 PE &  &  &  & $-$0.0003 \\ 
  &  &  &  & t = $-$4.219$^{***}$ \\ 
  & & & & \\ 
 Mkt.RF & $-$0.001 & $-$0.001 & $-$0.001 & $-$0.001 \\ 
  & t = $-$1.708$^{*}$ & t = $-$2.147$^{**}$ & t = $-$1.577 & t = $-$1.547 \\ 
  & & & & \\ 
 SMB & $-$0.004 & $-$0.004 & $-$0.004 & $-$0.004 \\ 
  & t = $-$6.575$^{***}$ & t = $-$6.191$^{***}$ & t = $-$6.656$^{***}$ & t = $-$6.596$^{***}$ \\ 
  & & & & \\ 
 HML & $-$0.001 & $-$0.001 & $-$0.001 & $-$0.001 \\ 
  & t = $-$1.836$^{*}$ & t = $-$2.286$^{**}$ & t = $-$1.679$^{*}$ & t = $-$1.801$^{*}$ \\ 
  & & & & \\ 
 RMW & 0.003 & 0.003 & 0.003 & 0.003 \\ 
  & t = 2.899$^{***}$ & t = 2.914$^{***}$ & t = 2.946$^{***}$ & t = 2.957$^{***}$ \\ 
  & & & & \\ 
 CMA & 0.001 & 0.002 & 0.001 & 0.001 \\ 
  & t = 1.138 & t = 1.614 & t = 0.836 & t = 0.978 \\ 
  & & & & \\ 
\hline \\[-1.8ex] 
Observations & 180 & 180 & 180 & 180 \\ 
R$^{2}$ & 0.435 & 0.477 & 0.482 & 0.488 \\ 
\hline 
\hline \\[-1.8ex] 
\textit{Note:}  & \multicolumn{4}{r}{$^{*}$p$<$0.1; $^{**}$p$<$0.05; $^{***}$p$<$0.01} \\ 
\end{tabular} 
\end{table} 

[Second table with different factor models, and PE.]

% Table created by stargazer v.5.2.2 by Marek Hlavac, Harvard University. E-mail: hlavac at fas.harvard.edu
% Date and time: to, maj 30, 2019 - 18:05:38
\begin{table}[!htbp] \centering 
  \caption{Robustnes to models} 
  \label{} 
\small 
\begin{tabular}{@{\extracolsep{5pt}}lccccc} 
\\[-1.8ex]\hline 
\hline \\[-1.8ex] 
 & \multicolumn{5}{c}{\textit{Dependent variable:}} \\ 
\cline{2-6} 
\\[-1.8ex] & \multicolumn{5}{c}{ESG\_deciles[, 11]} \\ 
\\[-1.8ex] & (1) & (2) & (3) & (4) & (5)\\ 
\hline \\[-1.8ex] 
 PE & $-$0.0004 & $-$0.0003 & $-$0.0003 & $-$0.0002 & $-$0.0003 \\ 
  & t = $-$4.443$^{***}$ & t = $-$4.117$^{***}$ & t = $-$4.223$^{***}$ & t = $-$2.097$^{**}$ & t = $-$4.219$^{***}$ \\ 
  & & & & & \\ 
 Mkt.RF &  & $-$0.002 & $-$0.001 & $-$0.001 & $-$0.001 \\ 
  &  & t = $-$6.159$^{***}$ & t = $-$2.924$^{***}$ & t = $-$1.836$^{*}$ & t = $-$1.547 \\ 
  & & & & & \\ 
 SMB &  &  & $-$0.005 & $-$0.005 & $-$0.004 \\ 
  &  &  & t = $-$7.541$^{***}$ & t = $-$8.237$^{***}$ & t = $-$6.596$^{***}$ \\ 
  & & & & & \\ 
 HML &  &  & $-$0.001 & 0.00001 & $-$0.001 \\ 
  &  &  & t = $-$1.464 & t = 0.023 & t = $-$1.801$^{*}$ \\ 
  & & & & & \\ 
 Mom &  &  &  & 0.001 &  \\ 
  &  &  &  & t = 4.191$^{***}$ &  \\ 
  & & & & & \\ 
 RMW &  &  &  &  & 0.003 \\ 
  &  &  &  &  & t = 2.957$^{***}$ \\ 
  & & & & & \\ 
 CMA &  &  &  &  & 0.001 \\ 
  &  &  &  &  & t = 0.978 \\ 
  & & & & & \\ 
 Constant & 0.007 & 0.007 & 0.006 & 0.002 & 0.005 \\ 
  & t = 2.407$^{**}$ & t = 2.838$^{***}$ & t = 2.846$^{***}$ & t = 1.082 & t = 2.321$^{**}$ \\ 
  & & & & & \\ 
\hline \\[-1.8ex] 
Observations & 180 & 180 & 180 & 180 & 180 \\ 
R$^{2}$ & 0.100 & 0.259 & 0.461 & 0.510 & 0.488 \\ 
\hline 
\hline \\[-1.8ex] 
\textit{Note:}  & \multicolumn{5}{r}{$^{*}$p$<$0.1; $^{**}$p$<$0.05; $^{***}$p$<$0.01} \\ 
\end{tabular} 
\end{table} 

[Third table with sentiment instead of PE. And regression of PE on sentiment.]

% Table created by stargazer v.5.2.2 by Marek Hlavac, Harvard University. E-mail: hlavac at fas.harvard.edu
% Date and time: fr, maj 31, 2019 - 17:37:17
\begin{table}[!htbp] \centering 
  \caption{Sentiment} 
  \label{} 
\small 
\begin{tabular}{@{\extracolsep{5pt}}lccc} 
\\[-1.8ex]\hline 
\hline \\[-1.8ex] 
 & \multicolumn{3}{c}{\textit{Dependent variable:}} \\ 
\cline{2-4} 
\\[-1.8ex] & \multicolumn{3}{c}{ESG\_deciles[c(13:180), 11]} \\ 
\\[-1.8ex] & (1) & (2) & (3)\\ 
\hline \\[-1.8ex] 
 PE[c(13:180)] & $-$0.0003 &  &  \\ 
  & t = $-$4.202$^{***}$ &  &  \\ 
  & & & \\ 
 Sustainability...United.States. &  & $-$0.0003 &  \\ 
  &  & t = $-$2.337$^{**}$ &  \\ 
  & & & \\ 
 climate...United.States. &  &  & $-$0.0002 \\ 
  &  &  & t = $-$1.925$^{*}$ \\ 
  & & & \\ 
 Mkt.RF[c(13:180)] & $-$0.001 & $-$0.001 & $-$0.001 \\ 
  & t = $-$1.640 & t = $-$1.587 & t = $-$1.767$^{*}$ \\ 
  & & & \\ 
 SMB[c(13:180)] & $-$0.004 & $-$0.004 & $-$0.004 \\ 
  & t = $-$6.681$^{***}$ & t = $-$6.372$^{***}$ & t = $-$6.512$^{***}$ \\ 
  & & & \\ 
 HML[c(13:180)] & $-$0.001 & $-$0.002 & $-$0.001 \\ 
  & t = $-$2.082$^{**}$ & t = $-$2.498$^{**}$ & t = $-$2.102$^{**}$ \\ 
  & & & \\ 
 RMW[c(13:180)] & 0.002 & 0.003 & 0.002 \\ 
  & t = 2.034$^{**}$ & t = 2.493$^{**}$ & t = 2.198$^{**}$ \\ 
  & & & \\ 
 CMA[c(13:180)] & 0.001 & 0.002 & 0.001 \\ 
  & t = 1.287 & t = 1.743$^{*}$ & t = 1.256 \\ 
  & & & \\ 
 Constant & 0.005 & 0.014 & 0.008 \\ 
  & t = 2.392$^{**}$ & t = 2.027$^{**}$ & t = 1.493 \\ 
  & & & \\ 
\hline \\[-1.8ex] 
Observations & 168 & 168 & 168 \\ 
R$^{2}$ & 0.485 & 0.447 & 0.441 \\ 
\hline 
\hline \\[-1.8ex] 
\textit{Note:}  & \multicolumn{3}{r}{$^{*}$p$<$0.1; $^{**}$p$<$0.05; $^{***}$p$<$0.01} \\ 
\end{tabular} 
\end{table} 

\end{comment}


\begin{table}[!htbp] \centering 
	\caption{} 
	\label{} 
	\begin{tabular}{@{\extracolsep{5pt}}lcccc} 
		\\[-1.8ex]\hline 
		\hline \\[-1.8ex] 
		& \multicolumn{4}{c}{\textit{Dependent variable:}} \\ 
		\cline{2-5} 
		\\[-1.8ex] & \multicolumn{4}{c}{LS[13:180]} \\ 
		\\[-1.8ex] & (1) & (2) & (3) & (4)\\ 
		\hline \\[-1.8ex] 
		gt[13:180] &  & $-$0.595 &  & 0.097 \\ 
		&  & (1.029) &  & (0.635) \\ 
		& & & & \\ 
		bt[13:180] &  & 0.052 &  & 0.784 \\ 
		&  & (0.705) &  & (0.477) \\ 
		& & & & \\ 
		Sustainability...United.States. & $-$0.008 & $-$0.002 &  &  \\ 
		& (0.010) & (0.011) &  &  \\ 
		& & & & \\ 
		climate...United.States. &  &  & $-$0.016$^{*}$ & $-$0.016$^{*}$ \\ 
		&  &  & (0.009) & (0.009) \\ 
		& & & & \\ 
		Mkt.RF[13:180] & $-$0.067$^{*}$ & $-$0.081$^{**}$ & $-$0.066$^{*}$ & $-$0.081$^{**}$ \\ 
		& (0.038) & (0.040) & (0.038) & (0.039) \\ 
		& & & & \\ 
		SMB[13:180] & $-$0.495$^{***}$ & $-$0.487$^{***}$ & $-$0.495$^{***}$ & $-$0.482$^{***}$ \\ 
		& (0.063) & (0.063) & (0.062) & (0.062) \\ 
		& & & & \\ 
		HML[13:180] & $-$0.002 & $-$0.008 & 0.002 & $-$0.011 \\ 
		& (0.060) & (0.060) & (0.058) & (0.059) \\ 
		& & & & \\ 
		Mom[13:180] & 0.169$^{***}$ & 0.157$^{***}$ & 0.172$^{***}$ & 0.156$^{***}$ \\ 
		& (0.034) & (0.035) & (0.033) & (0.035) \\ 
		& & & & \\ 
		Constant & 0.363 &  & 0.708 &  \\ 
		& (0.657) &  & (0.476) &  \\ 
		& & & & \\ 
		\hline \\[-1.8ex] 
		Observations & 168 & 168 & 168 & 168 \\ 
		Adjusted R$^{2}$ & 0.479 & 0.483 & 0.488 & 0.494 \\ 
		\hline 
		\hline \\[-1.8ex] 
		\textit{Note:}  & \multicolumn{4}{r}{$^{*}$p$<$0.1; $^{**}$p$<$0.05; $^{***}$p$<$0.01} \\ 
	\end{tabular} 
\end{table} 


\subsection*{Other climate sentiment measures}
We also test whether we should use other measures for climate sentiment in table X, and find that \emph{Climate} to be the best one.

\subsection*{Different sentiment measures}
In this section we consider sentiment measures separately for climate, socially responsible, and corporate governance. Table X shows that climate seems to be the main factor driving our ESG returns, closely followed by governance with social sentiment mattering the least.





\subsection{ESG Volatility}

This chapter investigates changes in ESG scores and their impact on returns. We test score volatility in the spirit of an event study, where we base the prediction on stock returns on factor loadings predicted from the time horizon of 18 months prior to the change in the score. Our event window lasts five months. For example, a firm that experiences a change in their ESG score from 2005 to 2006 of more than $X$ points is analyzed during the first 5 months of 2007. The estimation period runs from July 2004 until December 2006. Table~\ref{tab:eventstudy} and Figure~\ref{fig:eventstudy} display the results.

We find that drops of more than 15, 25, and 30 points in ESG scores lead to positive and significant cumulative abnormal returns. It is striking to see that, in particular, drops of more than 30 points lead to a cumulative abnormal return of more than 10\% over the time horizon of 5 months after the change. Increases in scores, however, do not yield the same effect. Even though we see significant negative cumulative abnormal returns for increases of more than 5, 10, 15, 20 points, they are minor in their magnitude. We find the findings rather counterintuitive, as we expected that negative changes would go hand in hand with negative abnormal returns, and vice versa. 

The drawback of this analysis clearly lies in the uncertainty of when investors receive access to the new ESG scores. The assumption herein is that ESG scores for the current year are available the last day of December of that particular year. It might be that investors anticipated or even received access to information to changes in the score throughout the current year and therefore adjusted their portfolios before our presumed event window. Furthermore, it could be that the markets exposure and therefore factor loadings change drastically over the event window. 

\begin{table}[!htbp] \centering 
	\caption{Event Study} 
	\fnote{The table exhibits the test statistics for the event study on changes in ESG scores. We report the number of events, the cumulative abnormal returns, the test statistics of $J_1$ and $J_2$ as well as the according $p$-values. We estimate factor loadings for every stock 18 months prior to the event up to one month before. The event window lasts five months. The post-event window includes another seven months.}
	\label{tab:eventstudy} 
	\begin{tabular}{@{\extracolsep{5pt}} ccccccc} 
		\\[-1.8ex]\hline 
		\hline \\[-1.8ex] 
		\multicolumn{1}{c}{$\Delta$ ESG} & \multicolumn{1}{c}{\# of Events} & \multicolumn{1}{c}{$\overline{CAR}$ (\%)} & \multicolumn{1}{c}{J$_1$} & \multicolumn{1}{c}{\textit{p}-value (\%)} & \multicolumn{1}{c}{J$_2$} & \multicolumn{1}{c}{\textit{p}-value (\%)} \\  
		\hline \\[-1.8ex] 
		$\leq$ -40 & 19 & 2.23 & 0.44 & 65.81 & 0.33 & 74.46 \\ 
		$\leq$ -35 & 39 & 9.20 & 1.51 & 13.18 & 2.14 & 3.22 \\ 
		\rowcolor{LightCyan} $\leq$ -30 & 73 & 10.72 & 2.78 & 0.55 & 3.67 & 0.02 \\ 
		\rowcolor{LightCyan} $\leq$ -25 & 157 & 6.09 & 2.68 & 0.73 & 2.48 & 1.31 \\ 
		$\leq$ -20 & 302 & 3.34 & 2.15 & 3.18 & 1.57 & 11.68 \\ 
		\rowcolor{LightCyan} $\leq$ -15  & 597 & 3.20 & 2.86 & 0.43 & 1.98 & 4.82 \\ 
		$\leq$ -10 & 1,154 & 1.63 & 2.13 & 3.29 & -0.19 & 85.25 \\ 
		$\leq$ -5 & 2,200 & 1.11 & 1.98 & 4.78 & -0.14 & 88.60 \\ 
		No Change & 1,109 & -0.01 & -0.02 & 98.79 & -0.85 & 39.71 \\ 
		\rowcolor{LightCyan} $\geq$ 5 & 3,039 & -1.13 & -2.37 & 1.79 & -2.99 & 0.28 \\ 
		\rowcolor{LightCyan} $\geq$ 10 & 1,915 & -1.72 & -2.85 & 0.44 & -3.19 & 0.14 \\ 
		\rowcolor{LightCyan} $\geq$ 15 & 1,238 & -1.30 & -1.79 & 7.28 & -2.51 & 1.20 \\ 
		\rowcolor{LightCyan} $\geq$ 20 & 809 & -1.82 & -2.06 & 3.94 & -2.87 & 0.41 \\ 
		$\geq$ 25 & 534 & -0.89 & -0.82 & 41.47 & -1.36 & 17.41 \\ 
		$\geq$ 30 & 351 & -0.58 & -0.44 & 65.73 & -1.22 & 22.39 \\ 
		$\geq$ 35 & 243 & -0.14 & -0.09 & 92.80 & -0.63 & 52.91 \\ 
		$\geq$ 40 & 161 & 0.28 & 0.16 & 87.60 & -0.35 & 72.77 \\ 
		\hline \\[-1.8ex] 
	\end{tabular} 
\end{table} 


\begin{figure}[!htbp]
	\centering
	\caption{Cumulative Abnormal Returns}
	\fnote{The plots exhibits the cumulative abnormal returns over the time line of the event study for different changes in the ESG score. -10 (10) takes into account all stocks that experienced a drop (increase) in their ESG score of 10 points or more and so on. $No Jump$ incorporates all stocks that experience a change of less than five ESG score points. We estimate factor loadings for every stock 18 months prior to the event up to one month before. The event window lasts five months. The post-event window includes another seven months.}
	\label{fig:eventstudy}
	\vspace{-1cm}
	\includegraphics[width=0.9\linewidth]{../figures/eventstudy.pdf}
\end{figure}


We can check changes in the market exposure by creating a trading strategy. If investors would indeed earn abnormal returns in the five months after significant drops in the score then they should by these stocks, hold them for a short period of time and then sell them again. We back-test this strategy by buying all stocks that were exposed to drops of more than 30 points, hold them for a year and then sell them again. Table~\ref{tab:tradingstrategy} reports the results. 

The findings show the opposite. We generate negative abnormal returns, which are only significant under CAPM. This suggests that stocks experiencing drops in their score quickly adjust their risk exposure. The market factor, for example, is relatively high with 1.418 under CAPM. 

We summarize the main findings of this chapter as follows. We find significant cumulative abnormal returns as a results of large drops in ESG scores and based on the market exposure estimated from before the change occurred. In an attempt to exploit these findings by setting up a trading strategy, we find that market exposure is quickly adjusted leaving us with negative abnormal returns. Nonetheless, this it seems as if ESG scores represent uncertainty for investors. Volatility in the scores impact returns. 



\begin{table}[!htbp] \centering 
	\caption{Trading Strategy} 
	\fnote{We buy stocks that experienced a drop of more than 30 points in their ESG score, hold them for a year, and then sell them again. We equally-weight our portfolio over all the stocks in year $t$. There are two years in which we do not hold any stocks are there are none that experienced changes of more than 30 points in their score. Our sample thereby drops from 180 to only 156 observations. $t$-values are reported in parenthesis. }
	\label{tab:tradingstrategy} 
	\begin{tabular}{@{\extracolsep{5pt}}lcccc} 
		\\[-1.8ex]\hline 
		\hline \\[-1.8ex] 
		& \multicolumn{4}{c}{\textit{Dependent variable:}} \\ 
		\cline{2-5} 
		\\[-1.8ex] & \multicolumn{4}{c}{Excess Return} \\ 
		\\[-1.8ex] & (1) & (2) & (3) & (4)\\ 
		\hline \\[-1.8ex] 
		alpha & \textbf{$-$0.574} & $-$0.454 & $-$0.356 & $-$0.415 \\ 
		& ($-$1.816) & ($-$1.512) & ($-$1.252) & ($-$1.317) \\ 
		& & & & \\ 
		Market & \textbf{1.418} & \textbf{1.301} & \textbf{1.215} & \textbf{1.284} \\ 
		& (18.303) & (15.932) & (15.292) & (14.294) \\ 
		& & & & \\ 
		SMB &  & 0.102 & 0.149 & 0.086 \\ 
		&  & (0.725) & (1.112) & (0.583) \\ 
		& & & & \\ 
		HML &  & \textbf{0.534} & \textbf{0.343} & \textbf{0.544} \\ 
		&  & (4.120) & (2.645) & (3.701) \\ 
		& & & & \\ 
		Mom &  &  & \textbf{$-$0.309} &  \\ 
		&  &  & ($-$4.440) &  \\ 
		& & & & \\ 
		RMW &  &  &  & $-$0.097 \\ 
		&  &  &  & ($-$0.440) \\ 
		& & & & \\ 
		CMA &  &  &  & $-$0.039 \\ 
		&  &  &  & ($-$0.158) \\ 
		& & & & \\ 
		\hline \\[-1.8ex] 
		Observations & 156 & 156 & 156 & 156 \\ 
		Adjusted R$^{2}$ & 0.683 & 0.715 & 0.747 & 0.712 \\ 
		\hline 
		\hline \\[-1.8ex] 
	\end{tabular} 
\end{table} 


\subsection{Trading strategy on Sentiment}
[Do this?]

\subsection{Decomposition into E S G}
[Make factor on E, S, and G seperately]

\subsection{Consideration of Demand and Supply effects}
Look at fund holdings as proxy for demand.
Look at aggregated ESG scores as proxy for supply of ESG.

\subsection{Robustness Checks and further Implications}

We conduct two test to check if our results are robust. First we subdivide in only five equally-weighted portfolios. Second, we additionally download information on the number of shares and the share prices from CRSP allowing us to value-weight our portfolios. In particular, we value-weight the portfolio in month $t$ based on market values in month $t-1$. 

When sorting our portfolios into only five portfolios instead of ten, we obtain similar results, see Appendix~\ref{app:robustness}. Portfolios P1 and P5 have a tendency to outperform others. However, significance for the long-short portfolio LS decreases. We see a similar picture when we condition on good and bad times. In bad times, ESG-constructed portfolios on the high and low end outperform others. In good times, however, ESG-sorted portfolios tend under-perform, which holds especially true to a significant level for the third quintile portfolio. 

Considering value-weighted ESG-sorted portfolios, we cannot confirm our previous findings. Except for a few exceptions, significance evaporates. We see this happening when looking at the entire time horizon as well as when conditioning on good and bad times. This suggests that our previous results are predominantly driven by small firms. We thereby conclude that ESG scores only have performance implications on small firms but are irrelevant for large firms.

\subsection{Event study using changes in ESG score on Thomson Reuters}


%We do the analysis for Quintile Portfolios.

\section{Conclusion}
This paper shows that ESG has become a factor that influences asset returns in good times i.e. investors do well, before they do good. Furthermore large drops in ESG, leads to an initial price drop and future increases in return. In the future it would be interesting to get more precise announcement times of the ESG scores to more cleanly isolate the ESG effects, from other correlated factors. It would also be interesting to look at whether the effects generally increased over time, as there has become more public interest. Looking at bonds and other markets would also be of interest, to see if this is an affect that holds there too.


\clearpage
\bibliographystyle{apa}
\bibliography{MyCollection}
%\bibliography{TheESGBenefitFactor}

\newpage
\begin{appendices}


\section{ESG Scores}
\label{app:esgscores}

\begin{figure}[!htp]
	\centering
	\caption{ESG Distribution}
	\fnote{Panel~\ref{fig:esg_scores} represents the distribution of ESG scores across all single yearly scores. Panel~\ref{fig:mean_esg_scores} averages the companies' yearly ESG scores.}
	\label{fig:esg_distribution}
	\begin{subfigure}{0.5\textwidth}
		\centering
		\caption{ESG Scores}
		\label{fig:esg_scores}
		\includegraphics[width=0.9\linewidth]{../figures/esg_scores.pdf}
	\end{subfigure}%
	\begin{subfigure}{0.5\textwidth}
		\centering
		\caption{Mean ESG Scores}
		\label{fig:mean_esg_scores}
		\includegraphics[width=0.9\linewidth]{../figures/mean_esg_scores.pdf}
	\end{subfigure}
\end{figure}


\begin{table}[!htbp] \centering 
	\caption{High Profile ESG Companies} 
	\fnote{The table exhibits companies of the highest decile ESG portfolio that were member of this prtfolio in both 2002 and 2016 (beginning and end of the sample). In total, we see 33 companies to be part of this group. The according CUSIP codes can be used to access the companies' information through CRSP.}
	\label{tab:high_esg_companies} 
	\begin{tabular}{@{\extracolsep{5pt}} clc} 
		\\[-1.8ex]\hline 
		\hline \\[-1.8ex] 
		\# & Name & CUSIP \\ 
		\hline \\[-1.8ex] 
		1 & A B B LTD & 00037520 \\ 
		2 & ABBOTT LABORATORIES & 00282410 \\ 
		3 & BANCO BILBAO VIZCAYA ARGENTARIA & 05946K10 \\ 
		4 & BANCO SANTANDER CENTRAL HISP SA & 05964H10 \\ 
		5 & BAXTER INTERNATIONAL INC & 07181310 \\ 
		6 & B H P LTD & 08860610 \\ 
		7 & BOEING CO & 09702310 \\ 
		8 & BRISTOL MYERS SQUIBB CO & 11012210 \\ 
		9 & BRITISH AMERICAN TOBACCO PLC & 11044810 \\ 
		10 & CHEVRON CORP & 16676410 \\ 
		11 & CISCO SYSTEMS INC & 17275R10 \\ 
		12 & DOW CHEMICAL CO & 26054310 \\ 
		13 & DU PONT E I DE NEMOURS \& CO & 26353410 \\ 
		14 & DUKE ENERGY CORP & 26441C20 \\ 
		15 & EASTMAN CHEMICAL CO & 27743210 \\ 
		16 & ENBRIDGE INC & 29250N10 \\ 
		17 & GLAXOSMITHKLINE PLC & 37733W10 \\ 
		18 & HEWLETT PACKARD CO & 40434L10 \\ 
		19 & IMPERIAL OIL LTD & 45303840 \\ 
		20 & I N G GROEP N V & 45683710 \\ 
		21 & INTEL CORP & 45814010 \\ 
		22 & INTERNATIONAL BUSINESS MACHS COR & 45920010 \\ 
		23 & JOHNSON \& JOHNSON & 47816010 \\ 
		24 & KONINKLIJKE PHILIPS ELEC N V & 50047230 \\ 
		25 & MERCK \& CO INC & 58933Y10 \\ 
		26 & MOTOROLA INC & 62007630 \\ 
		27 & NOKIA CORP & 65490220 \\ 
		28 & OCCIDENTAL PETROLEUM CORP & 67459910 \\ 
		29 & PROCTER \& GAMBLE CO & 74271810 \\ 
		30 & STMICROELECTRONICS NV & 86101210 \\ 
		31 & TEXAS INSTRUMENTS INC & 88250810 \\ 
		32 & MINNESOTA MINING \& MFG CO & 88579Y10 \\ 
		33 & UNITED PARCEL SERVICE INC & 91131210 \\ 
		\hline \\[-1.8ex] 
	\end{tabular} 
\end{table} 


\clearpage
\section{ESG Portfolios}
\label{app:ESGportfolios}


\begin{figure}[!htp]
	\centering
	\caption{Raw Returns}
	\fnote{The plots \ref{fig:rawreturns_ew} and \ref{fig:rawreturns_vw} exhibit the decile portfolio raw return. The high (low) ESG decile portfolio \textit{1} (\textit{10}) depicts the firms with the highest (lowest) ESG scores. Portfolios are rearranged every year according to the previous year's ESG score.}
	\label{fig:rawreturns}
	\begin{subfigure}{0.5\textwidth}
		\centering
		\caption{Equally-weighted}
		\label{fig:rawreturns_ew}
		\includegraphics[width=0.9\linewidth]{../figures/portfoliosrawreturn_ew.pdf}
	\end{subfigure}%
	\begin{subfigure}{0.5\textwidth}
		\centering
		\caption{Value-weighted}
		\label{fig:rawreturns_vw}
		\includegraphics[width=0.9\linewidth]{../figures/portfoliosrawreturn_vw.pdf}
	\end{subfigure}
\end{figure}


\begin{figure}[!htp]
	\centering
	\caption{ESG Portfolio Returns}
	\fnote{The figure plots the highest and lowest equally-weighted decile cumulative return portfolio. The high (low) ESG decile portfolio 1. D. (10. D.) depicts the highest (lowest) firms with the highest (lowest) ESG scores. Portfolios are rearranged every year according to the previous year's ESG score. Moreover, the plot shows the equally-weighted ESG portfolio, a portfolio that includes every firm with a score, regardless of its magnitudes. Also, all firms' returns with no ESG scores in the sample are equally-weighted and combined in a portfolio. Finally, the S\&P500 serves as a comparable time series.}
	\label{fig:esg_returns}
	%	\vspace{-0.5cm}
	\includegraphics[width=0.7\linewidth]{../figures/cumreturndec_ew.pdf}
\end{figure}



\begin{figure}[!htp]
	\centering
	\caption{Size Distribution}
	\fnote{The figure plots the size distribution over the equally-weighted ESG-portfolios. The high (low) ESG decile portfolio 1. D. (10. D.) depicts the highest (lowest) firms with the highest (lowest) ESG scores. Portfolios are rearranged every year according to the previous year's ESG score.}
	\label{fig:sizedistr}
	%	\vspace{-0.5cm}
	\includegraphics[width=0.7\linewidth]{../figures/sizedistribution.pdf}
\end{figure}


\clearpage
\section{Good versus Bad times}

\begin{figure}[!htp]
	\centering
	\caption{Good and Bad times}
	\fnote{}
	\label{fig:goodversusbaddummy}
	\begin{subfigure}{0.5\textwidth}
		\centering
		\caption{Good times}
		\label{fig:goodtimesdummy}
		\includegraphics[width=0.9\linewidth]{../figures/goodtimes.pdf}
	\end{subfigure}%
	\begin{subfigure}{0.5\textwidth}
		\centering
		\caption{Bad times}
		\label{fig:badtimesdummy}
		\includegraphics[width=0.9\linewidth]{../figures/badtimes.pdf}
	\end{subfigure}
\end{figure}




\clearpage
\section{Robustness Checks}
\label{app:robustness}


\begin{table}[!htbp] \centering 
	\caption{Risk-adjusted and value-weighted ESG portfolio returns with badCAPEtimes Dummy} 
	\fnote{The returns of the value-weighted ESG portfolios are risk adjusted through the application of the CAPM, 3-factor, 4-factor, and 5-factor models. The LS portfolio goes long in the highest ESG decile firms and shorts the lowest ESG decile firms. Additionally, we condition the regression on good and bad times through binary variables as in $r_{it} - r_t^f = \alpha_i^G G_t + \alpha_i^B B_t + + \sum_{j=1}^{n} \beta_{ij} f_{jt} + \epsilon_{it}$. Here, $\alpha^G$ exhibits the abnormal return in good times, whereas $\alpha^B$ depicts the abnormal return that is additionally earned in bad times. Good times (bad times) are defined by the current P/E ratio being above (below) the current price over the 10-year rolling average earnings. The tables depicts alphas in percentage points. $t-values$ test if the estimated abnormal returns are significantly different from zero and bold numbers signal significance to the 10\% level or less.}
	\label{tab:goodandbad_value} 
	\resizebox{\textwidth}{!}{\begin{tabular}{@{\extracolsep{5pt}} lccccccccccc}
			\\[-1.8ex]\hline 
			\hline \\[-1.8ex] 
			 & P1 & P2 & P3 & P4 & P5 & P6 & P7 & P8 & P9 & P10 & LS \\ 
			\hline \\[-1.8ex] 
			CAPM alpha$^G$ & \textbf{0.513} & \textbf{0.728} & \textbf{1.068} & \textbf{0.766} & \textbf{0.888} & \textbf{0.891} & \textbf{0.578} & \textbf{1.179} & \textbf{0.914} & \textbf{2.054} & -0.084 \\ 
			t-value & 2.278 & 2.709 & 3.292 & 2.37 & 2.893 & 2.762 & 1.819 & 2.876 & 2.214 & 4.763 & -0.171 \\[2ex]
			
			CAPM alpha$^B$ & 0.033 & 0.176 & -0.018 & -0.074 & -0.148 & \textbf{-0.223} & -0.17 & -0.102 & -0.091 & -0.114 & -0.056 \\ 			
			t-value & 0.353 & 1.585 & -0.138 & -0.556 & -1.173 & -1.679 & -1.297 & -0.606 & -0.533 & -0.641 & -0.278 \\[4ex] 
			
			3-factor alpha$^G$ & \textbf{0.596} & \textbf{0.765} & \textbf{1.039} & \textbf{0.587} & \textbf{0.675} & \textbf{0.634} & 0.302 & \textbf{0.841} & 0.554 & \textbf{1.755} & 0.343 \\ 
			t-value & 2.626 & 2.814 & 3.283 & 1.903 & 2.413 & 2.154 & 1.082 & 2.367 & 1.517 & 4.435 & 0.736 \\[2ex] 
			
			3-factor alpha$^B$ & 0.025 & 0.178 & 0.004 & -0.036 & -0.103 & -0.175 & -0.116 & -0.031 & -0.022 & -0.052 & -0.105 \\ 			
			t-value & 0.267 & 1.613 & 0.029 & -0.288 & -0.905 & -1.459 & -1.02 & -0.216 & -0.145 & -0.32 & -0.552 \\[4ex] 
			
			4-factor alpha$^G$ & 0.375 & 0.436 & 0.446 & 0.001 & 0.105 & 0.066 & -0.24 & 0.019 & -0.11 & \textbf{1.019} & \textbf{0.826} \\ 
			t-value & 1.622 & 1.603 & 1.517 & 0.004 & 0.414 & 0.242 & -0.934 & 0.063 & -0.32 & 2.763 & 1.747 \\[2ex] 
			
			4-factor alpha$^B$ & 0.082 & 0.263 & 0.156 & 0.115 & 0.044 & -0.028 & 0.024 & 0.18 & 0.149 & 0.138 & -0.229 \\ 
			t-value & 0.374 & 0.016 & 0.182 & 0.313 & 0.664 & 0.792 & 0.816 & 0.139 & 0.273 & 0.346 & 0.223 \\[4ex] 
			
			5-factor alpha$^G$ & \textbf{0.647} & \textbf{0.777} & \textbf{1.07} & \textbf{0.597} & \textbf{0.76} & \textbf{0.694} & 0.43 & \textbf{0.936} & \textbf{0.67} & \textbf{1.955} & 0.023 \\ 
			t-value & 2.806 & 2.808 & 3.335 & 1.9 & 2.688 & 2.327 & 1.553 & 2.601 & 1.822 & 5.02 & 0.052 \\[2ex] 
			
			5-factor alpha$^B$ & 0.038 & 0.158 & -0.026 & -0.016 & -0.078 & -0.194 & -0.124 & -0.014 & 0.041 & 0.062 & -0.271 \\ 
			t-value & 0.399 & 1.385 & -0.197 & -0.121 & -0.671 & -1.578 & -1.086 & -0.096 & 0.268 & 0.385 & -1.475 \\ 
			\hline \\[-1.8ex] 
	\end{tabular}}
\end{table} 







\end{appendices}

\end{document}
